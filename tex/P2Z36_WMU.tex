\documentclass[9pt]{beamer}
\usetheme{Warsaw}
\usefonttheme[onlymath]{serif}

\usepackage[utf8]{inputenc}
%\usepackage[polish]{babel}
\usepackage[T1]{fontenc}
\usepackage{comment}
\usepackage{setspace}
\usepackage{amsmath}
\usepackage{dcolumn}
\usepackage{siunitx}
\usepackage{tabularx}
\usepackage{graphicx}
\usepackage{changepage}
\usepackage[backend=biber,style=numeric,sorting=nyt,language=polish]{biblatex} % Konfiguracja biblatex
\addbibresource{references.bib} % Link the .bib file

\sisetup{
    table-number-alignment = center, % Align numbers at the decimal point
    table-format = +1.4e3,           % Specify number format: 1 digit before, 3 after decimal, and 1 exponent
    tight-spacing = true,           % Remove extra spacing around numbers
}

\newcolumntype{M}{>{\centering\arraybackslash\math}p{2cm}}

\linespread{1}

%\newcommand\eqozn{\stackrel{\mathclap{\normalfont\mbox{\normalfont\tiny ozn}}}{=}}
\newcommand\eqozn{\mathrel{\overset{\makebox[0pt]{\mbox{\scriptsize ozn}}}{=}}}

\title{Projekt 2, Zadanie 36}
\author{Wiktor Murawski, 333255, grupa 3, środa 12:15}
\date{}

\begin{document}

\begin{frame}
    %\frametitle{\insertauthor,\space\inserttitle}
    \begin{spacing}{1.75}
    \begin{center}
        \inserttitle\par
        \insertauthor
    \end{center}
    \vspace{2em}
    Metoda Adamsa-Bashfortha rzędu 4-go dla liniowych równań różniczkowych pierwszego i drugiego rzędu. Wartości początkowe $y_1$, $y_2$, $y_3$ obliczane metodą Rungego-Kutty rzędu 4-go (wzór Ralstona). 
    \end{spacing}
\end{frame}

\section{Wprowadzenie}
\begin{frame}{Równanie różniczkowe pierwszego rzędu}
    Dane jest równanie różniczkowe liniowe pierwszego rzędu oraz warunek początkowy:
    \begin{equation*}
        \label{eq1}
        a_1(x)y' + a_0(x)y = b(x), \qquad y(x_0) = y_0
    \end{equation*}
    %na przedziale $\begin{bmatrix} x_0, x_N \end{bmatrix}$ 

    \hspace{2cm}
    
    Przekształcając równanie otrzymujemy
    \begin{equation*}
        y' = f(x,y) = \frac{b(x) - a_0(x)y}{a_1(x)}
    \end{equation*}
    
\end{frame}

\begin{frame}{Równanie różniczkowe drugiego rzędu}
    Dane jest równanie różniczkowe liniowe drugiego rzędu oraz warunki początkowe:
    \begin{equation*}
        \label{eq2}
        a_2(x)y'' + a_1(x)y' + a_0(x)y = b(x), \qquad \begin{matrix}y(x_0) = y_0 \\ y'(x_0) = y'_0\end{matrix}
    \end{equation*}
    %na przedziale $\begin{bmatrix} x_0, x_N \end{bmatrix}$

    Sprowadzamy równanie do układu równań różniczkowych liniowych stopnia pierwszego: 
    \[ Y' = F(x,Y), \] gdzie
    \begin{equation*}
        Y = \begin{bmatrix} y \\ y' \end{bmatrix} \eqozn \begin{bmatrix} y_1 \\ y_2 \end{bmatrix} \qquad\qquad Y' = \begin{bmatrix} y_1' \\ y_2' \end{bmatrix}
    \end{equation*}

    Przekształacając równanie otrzymujemy
    \begin{equation*}
        Y' = F(x,Y) = \begin{bmatrix} y_2 \\ \dfrac{1}{a_2(x)}(b(x) - a_0y_1 - a_1y_2) \end{bmatrix}
    \end{equation*}
    
\end{frame}

\section{Metoda Rungego-Kutty (Ralston)}
\begin{frame}{Jawne metody Rungego-Kutty 4-go rzędu}
    Mając wartość $Y_n$, jawne metody Rungego-Kutty rzędu czwartego pozwalają na wyznaczenie wartości $Y_{n+1}$ następująco:
    $$ \begin{aligned}
        K_1 &= hF(x_n, Y_n) \\
        K_2 &= hF(x_n + a_2h, Y_n + b_{21}K_1) \\
        K_3 &= hF(x_n + a_3h, Y_n + b_{31}K_1 + b_{32}K_2) \\
        K_4 &= hF(x_n + a_4h, Y_n + b_{41}K_1 + b_{42}K_2 + b_{43}K_3) \\
        Y_{n+1} - Y_n &= c_1K_1 + c_2K_2 + c_3K_3 + c_4K_4
    \end{aligned} $$
    Współczynniki $a_i, b_{ij},c_i$ przedstawione w tablicy Butchera:
    $$\begin{array}{c|cccc}
    0 & 0 & 0 & 0 & 0 \\
    a_2 & b_{21} & 0 & 0 & 0 \\
    a_3 & b_{31} & b_{32} & 0 & 0 \\
    a_4 & b_{41} & b_{42} & b_{43} & 0 \\ \hline
    & c_1 & c_2 & c_3 & c_4
    \end{array}$$
\end{frame}

\begin{frame}{Ogólne wzory na współczynniki}
    \newcommand{\1}{\alpha}
    \newcommand{\2}{\beta}

    \small
    
    Współczynniki $a_i, b_{ij},c_i$ tworzą rodzinę dwuparametrową zależną od parametrów $\1$ i $\2$ gdzie $\1 \neq 0, \2 \neq 0, \1 \neq 1, \2 \neq 1, \1 \neq \2$.

    \normalsize
    \renewcommand{\arraystretch}{2.5}
    \begin{adjustwidth}{-2.5em}{-3em}
    \scalebox{0.55}{$
    \begin{array}{c|ccccc}
    0 & 0 & 0 & 0 & 0  \\
    \1 & \1 & 0 & 0 & 0  \\
    \2 & \2 - \dfrac{\2(\2-\1)}{2\1(1-2\1)} & \dfrac{\2(\2-\1)}{2\1(1-2\1)} & 0 & 0  \\
    1 & 
1 - \dfrac{(1-\1)(\2(\1+\2-1-(2\2-1)^2) + 2\1(1-2\1)(1-\2))}{2\1\2(\2-\1)(6\1\2-4(\1+\2)+3)}
    
% \dfrac
% {12\1^2\2^2 + 12\1^2\2 - 4\1^2 + 12\1\2^2 - 15\1\2 + 6\1 - 4\2^2 + 5\2 - 2}
% {2\1\2(4\1 + 4\2 - 6\1\2 - 3)}
& \dfrac{(1-\1)(\1+\2-1-(2\2-1)^2)}{2\1(\2-\1)(6\1\2-4(\1+\2)+3)} & \dfrac{(1-2\1)(1-\1)(1-\2)}{\2(\2-\1)(6\1\2-4(\1+\2)+3)} & 0  \\ \hline
      & \dfrac{1}{2}-\dfrac{1-2(\1+\2)}{12\1\2} & \dfrac{2\2 - 1}{12\1(\2-\1)(1-\1)} & \dfrac{1 - 2\1}{12\2(\2-\1)(1-\2)} & \dfrac{1}{2}+\dfrac{2(\1+\2)-3}{12(1-\1)(1-\2)}
    \end{array}
    $}
    \end{adjustwidth}

    \vspace{0.5cm}

    \small
    
    Dla parametrów $\alpha = \frac{2}{5} \text{ i } \beta = \frac{7}{8} - \frac{3\sqrt5}{16}$ otrzymujemy ograniczenie górne błędu:
    \[ |E| < 5.46\cdot10^{-2}ML^4\]
    gdzie, dla pewnego obszaru $B(x,y)$ zawierającego $(x_n,y_n)$, zachodzi
    $$f(x,y) \leq M \qquad\qquad \dfrac{\partial^{i+j}f}{\partial x^i\partial y^j} < \dfrac{L^{i+j}}{M^{j-1}} $$
    
\end{frame}

\begin{frame}{Tabela Butchera (wartości dokładne)}
    % Wyznaczone wartości współczynników dla parametrów minimalizujących ograniczenie górne błędu ($\alpha = \frac{2}{5} \text{ i } \beta = \frac{7}{8} - \frac{3\sqrt5}{16}$). Dla tych $\alpha$ i $\beta$:

    % \vspace{16pt}
    \renewcommand{\arraystretch}{2.5}
    \scalebox{0.80}{$
    \begin{array}{c|cccc}
    0 & 0 & 0 & 0 & 0 \\
    \dfrac{2}{5}            & \dfrac{2}{5}                          & 0                                     & 0                                                 & 0 \\
    \dfrac{14-3\sqrt5}{16}  & \dfrac{-2\,889 + 1\,428\sqrt5}{1\,024}   & \dfrac{3\,785 - 1\,620\sqrt5}{1\,024}    & 0                                                 & 0 \\
    1                       & \dfrac{-3\,365 + 2\,094\sqrt5}{6\,040}   & \dfrac{-975 - 3\,046\sqrt5}{2\,552}     & \dfrac{467\,040 + 203\,968\sqrt5}{240\,845}          & 0 \\ \hline
                            & \dfrac{263 + 24\sqrt5}{1\,812}         & \dfrac{125 - 1\,000\sqrt5}{3\,828}       & \dfrac{3\,426\,304 + 1\,661\,952\sqrt5}{5\,924\,787}    & \dfrac{30 - 4\sqrt5}{123}
    \end{array}
    $}
\end{frame}

\begin{frame}{Tabela Butchera (wartości przybliżone)}
    \renewcommand{\arraystretch}{1.5} % Adjust row spacing for better readability
    % \scalebox{1}{$
    % \begin{array}{c|ccccc}
    % 0 & 0 & 0 & 0 & 0  \\
    % 0.4 & 0.4 & 0 & 0 & 0  \\
    % 0.45573725 & 0.29697761 & 0.15875964 & 0 & 0  \\
    % 1 & 0.21810040 & -3.05096516 & 3.83286476 & 0  \\ \hline
    %   & 0.17476028 & -0.55148066 & 1.20553560 & 0.17118478
    % \end{array}
    % $}
    \scalebox{1}{$
    \begin{array}{c|ccccc}
    0 & 0 & 0 & 0 & 0  \\
    0.4 & 0.4 & 0 & 0 & 0  \\
    0.45573725 & 0.29697761 & 0.15875964 & 0 & 0  \\
    1 & 0.21810039 & -3.05096515 & 3.83286476 & 0  \\ \hline
      & 0.17476028 & -0.55148066 & 1.20553560 & 0.17118478
    \end{array}
    $}
\end{frame}

\section{Metoda Adamsa-Bashfortha}
\begin{frame}{Metoda Adamsa-Bashfortha}
    
\end{frame}

\section{Testy}
\begin{frame}
	\frametitle{Testy poprawności}
\end{frame}


\begin{frame}
	\frametitle{Testy numeryczne}
	\begin{spacing}{1.25}
		Przetestujemy teraz własności numeryczne zaimplementowanej metody. 
		% Zaobserwujemy jak metoda działa dla kilku wybranych funkcji, które nie są wielomianami stopnia $ < 2 $.\\
		% Zauważymy, że dla niskich $n$ wyniki są bardzo niedokładne. Intuicyjnie, im wyższe n, tym wyniki będą dokładniejsze; $k$-krotne zwiększenie wartości $n$ spowoduje około $k^2$-krotne zmniejszenie wartości błędu bezwzględnego.\\
		% Wynika to z faktu wspomnianego na wykładzie, mianowicie jeśli $f$ jest wystarczająco wiele razy różniczkowalna, to 
		% $$ | S^{[n]}(f) - I(f) | = \mathcal{O}(n^{-p})$$
		% gdzie $p$ jest rzędem zastosowanej kwadratury, w naszym przypadku 
		% $$ | S_S^{[n]}(f) - I(f) | = \mathcal{O}(n^{-2}) $$
	\end{spacing}
\end{frame}



\begin{frame}
	\frametitle{Testy numeryczne}
\end{frame}


\begin{frame}
	\frametitle{Źródła}
    \nocite{*}
    \printbibliography
\end{frame}


\end{document}


% @article{Ralston1962RungeKuttaMW,
%   title={Runge-Kutta methods with minimum error bounds},
%   author={Anthony Ralston},
%   journal={Mathematics of Computation},
%   year={1962},
%   volume={16},
%   pages={431-437},
%   url={https://www.ams.org/journals/mcom/1962-16-080/S0025-5718-1962-0150954-0/S0025-5718-1962-0150954-0.pdf},
% }